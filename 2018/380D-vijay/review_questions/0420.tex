\section{April 20 Review Questions}
\begin{QandA}
   \item How does the size of a task affect MapReduce?
         \begin{answered}
         MapReduce takes a user-submitted job and maps to $M$ tasks and then reduce to $R$ tasks. As shown in the paper, the master
         must make $O(M + R)$ scheduling decisions and keeps $O(M \ast R)$ state in memory. Thus, given the job size fixed, the smaller
         the task in each phase (map phase or reduce phase), the larger the variable is ($M$ or $R$). Accordingly, the master will have
         more scheduling decisions to make and keep more state in memory. However, if the task size is too big, then the advantage of 
         parallel execution on a large cluster offered by MapReduce goes in vain.
         \end{answered}

   \item Spark uses a very different mechanism for fault tolerance than most systems we've studied so far. What's one assumption that allows Spark to use this method?
         \begin{answered}
         Spark uses a distributed memory abstraction called Resilient Distributed Datasets (RDDs), which is a read-only, partitioned collections
         of records to achieve the fault tolerance. Since RDDs are immutable, then we can effectively recompute the faulty RDDs given the initial
         RDDs and the lineage graph that is from initial RDDs to the faulty one.
         \end{answered}
   \item Describe one way in which a datacenter has different needs or limitations than a single machine or small cluster.
         \begin{answered}
         Compared to a single machine, datacenter has disadvantage in terms of both latency and bandwidth. Usually, a single machine can 
         access its DRAM in 100 ns. However, for machines on rack to access other machine's DRAM in the same rack, the latency can take
         300 $\mu$s. Similarly, machine that accesses its local DRAM has bandwidth 20 GB/s. However, for the same machine to access other
         machine's DRAM within the same rack, bandwidth drops to 100 MB/s. When compared to small cluster, datacenter has to worry about
         the enery consumption, indoor humidity, and indoor temperature. In addition, the probability of a single machine failure at any time
         can become more noticeable in a datacenter than in a small cluster.
         \end{answered}
\end{QandA}




