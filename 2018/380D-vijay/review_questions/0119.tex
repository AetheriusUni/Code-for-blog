\section{Jan 19 Review Questions}
\begin{QandA}
   \item How do the 8 fallacies of distributed systems differ between designing a system over the world wide web (say between India and the US) and between servers inside a single datacenter? 
         \begin{answered}
         \begin{enumerate}
          \item The network is reliable. The network for a system that spans globablly may be highly likely unreliable. However, for a system
          resides only in a single datacenter. We can think the network is reliable. However, there are still chances for network outage even
          in a single datacenter.
          \item Latency is zero. The latency for a global system is necessarily non-zero and large. However, for servers in a single datacenter, we can assume that the latency is very small but we still cannot assume that the latency is zero.
          \item Bandwidth is infinite. No matter which scenarios, the bandwidth is limited.
          \item The network is secure. For servers within a single datacenter, if we have built a good firewall to isolate the network inside the
          datacenter from outside, we are likely to have a secure network. However, for a system that relies on a global network, we need to put
          much more effort to ensure the security of the network.
          \item Topology doesn't change. We can assume this one for servers in a single network. However, for a system that utilizes multiple datacenters, the topology may change over the time because we may switch to other datacenters during the time.
          \item There is one administrator. In either scenarios, we cannot assume we have one administrator. This is especially true for a global
          sytem that needs to provide a 24x7 service.
          \item Transport cost is zero. Similar to the latency, we cannot assume the transport cost is zero in either cases.
          \item The network is homogeneous. For servers in a single datacenter, we can assume this one holds. However, for a global system, this
          assumption can never be true because the network infrastructure is necessarily different from country to country.
         \end{enumerate}
         \end{answered}

   \item What would you expect to be different between Tandem's breakdown of outage reasons and the breakdown of outage reasons for a modern service, say Amazon EC2?
         \begin{answered}
         I think the administration outages still are the major source of the outage for both Tandem and a modern service given that 
         the notable outages from last year (\fnurl{GitLab}{https://about.gitlab.com/2017/02/10/postmortem-of-database-outage-of-january-31/} and \fnurl{AWS}{https://aws.amazon.com/message/41926/}) are due to the human errors. In addition, the software outage may also take the second
         place for the outage reasons given the complexity of the system has increased. However, I would expect that the outage reasons due to the
         hardware and environment may no longer hold for a modern service nowadays. The stability of hardware has increased since Tandem's paper
         and for a large scare sevice like Amazon EC2, people use multiple datacenters to provide redundancy instead of multiple servers reside in
         the same datacenter. Providing service backup using multiple datacenters across different locations make the outage due to power, communications, and facilities very unlikely.
         \end{answered}
\end{QandA}

