\section{HW1}
Translate the following sentences into predicate logic. 	
(You may choose your own letters to serve as non-logical constants. Translate ``but'' as if it were ``and'')
\begin{QandA}
   \item John loves Mary, but she doesn’t love him. 
         \begin{answered}
		 $\texttt{love}(J,M) \land \neg \texttt{love}(M,J)$
         \end{answered}

   \item John believes all things that Mary believes and some other things as well.
         \begin{answered}
         $(\forall y)((\texttt{things}(y) \land \texttt{believe}(M,y)) \rightarrow \texttt{believe}(J,y)) \land (\exists x)(\texttt{things}(x)
         \land \texttt{believe}(J,x) \land \neg \texttt{believe}(M,x))$
         \end{answered}
         
   \item If a cat and a mouse are in the same room and the mouse doesn’t run away, then either the mouse is dead or the cat is dead.
         \begin{answered}
         $(\forall x)(\forall y)(\texttt{cat}(x) \land \texttt{mouse}(y) \land \texttt{inTheSameRoom}(x,y) \land \neg \texttt{runAway}(y)) 
         \rightarrow (\texttt{dead}(x) \lor \texttt{dead}(y))$ \footnote{Here, I interpret ``either ... or" as $\texttt{OR}$ in logic term. If
         we treat ``either ... or" as $\texttt{XOR}$ instead, the PC of the sentence then becomes
         $(\forall x)(\forall y)(\texttt{cat}(x) \land \texttt{mouse}(y) \land \texttt{inTheSameRoom}(x,y) \land \neg \texttt{runAway}(y)) 
         \rightarrow ((\texttt{dead}(x) \land \neg \texttt{dead}(y)) \lor (\texttt{dead}(y) \land \neg \texttt{dead}(x)))$} 
         \end{answered}
   \item Everyone who has two jobs neglects one of them.
         \begin{answered}
         $(\forall x)(\exists y)(\exists z)(\texttt{Human}(x) \land \texttt{job}(y) \land \texttt{job}(z) \land \texttt{hasJob}(x,y) \land
         \texttt{hasJob}(x,z)) \rightarrow ((\texttt{neglect}(y) \land \neg \texttt{neglect}(z)) \lor (\texttt{neglect}(z) \land 
         \neg \texttt{neglect}(y)))$
         \end{answered}
   \item If an argument with two premises is valid and its conclusion is false, then one of the premises is false.
         \begin{answered}
         $(\forall x)(\forall y)(\forall z)(\forall k)((\texttt{argument}(x) \land \texttt{premise}(y) \land \texttt{premise}(z)
         \land \texttt{belongToArgument}(y,x) \land \texttt{belongToArgument}(z,x) \land \texttt{valid}(x) \land \texttt{conclusion}(k)
         \land \neg \text{conclusionIsTrue}(k) \land \texttt{belongToArgument}(k,x)) \rightarrow (\neg y \lor \neg z))$ 
         \footnote{Since an argument is consisted of premises and conclusion, $\texttt{conclusion}(k)$ may be ambiguous as it can either
         represent whether $k$ is a conclusion or not in an argument (it can be premise) or it can represent the truth value of a conclusion. Thus,
         I think it is necessary to avoid this ambiguity by introducting another predicate $\texttt{conclusionIsTrue}$ to indicate the
         truth value of a conclusion and use $\texttt{conclusion}$ predicate to indicate whether the given variable is a conclusion or not.}
         \end{answered}
   \item There are three dishes that John doesn’t like.
         \begin{answered}
         $((\exists x)(\exists y)(\exists z)(\texttt{dish}(x) \land \texttt{dish}(y) \land \texttt{dish}(z))) \rightarrow (
         \neg \texttt{like}(J,x) \land \neg \texttt{like}(J,y) \land \neg \texttt{like}(J,z))$
         \end{answered}
\end{QandA}